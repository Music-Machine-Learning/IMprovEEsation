\begin{thebibliography}{50}
% XXX TUTTI ASSIEME XXX %
  \bibitem{Simpson} Homer J. Simpson. \textsl{Mmmmm...donuts}.
		Evergreen Terrace Printing Co., Springfield, SomewhereUSA, 1998
  \bibitem{yt:evol} Murali S. N., \textsl{Music: Markov Chains, Genetic Algorithms and Scale Transformation}.
		on Youtube at https://www.youtube.com/watch?v=FvrcvXpcfvM, 2013
  \bibitem{biab} Gannon, P., \textsl{Band-in-a-Box}. PG Music, 1990
  \bibitem{evomus} Miranda E. R., Biles J. A., \textsl{Evolutionary Computer Music}, Springer Ed., 2007
  \bibitem{gendash} WASCHKA II, R. O. D. N. E. Y. "Composing with Genetic Algorithms: GenDash." Evolutionary Computer Music. Springer London, 2007. 117-136.
  \bibitem{genjam} Biles, John. "GenJam: A genetic algorithm for generating jazz solos." Proceedings of the International Computer Music Conference. INTERNATIONAL COMPUTER MUSIC ACCOCIATION, 1994.
  \bibitem{hmm}  Jan Pavelka; Gerard Tel; Miroslav Bartosek, eds. (1999). Factor oracle: a new structure for pattern matching; Proceedings of SOFSEM’99; Theory and Practice of Informatics. Springer-Verlag, Berlin. pp. 291–306. ISBN 3-540-66694-X. Retrieved 4 December 2013. "Lecture Notes in Computer Science 1725"
  \bibitem{dubnov} S. Dubnov, G. Assayag, O. Lartillot, G. Bejerano, "Using Machine-Learning Methods for Musical Style Modeling", IEEE Computers, 36 (10), pp. 73-80, Oct. 2003.
  \bibitem{cont} Pachet, Francois The Continuator: Musical Interaction with Style. In ICMA, editor, Proceedings of ICMC, pages 211-218, Göteborg, Sweden, September 2002 ICMA. best paper award 
  \end{thebibliography}
