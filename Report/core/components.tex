\section{Componenti del Sistema}
% XXX BERA XXX %
Come indicato in precedenza, le principali componenti a stretto contatto
tra loro sono \emph{direttore}, \emph{musicisti} e \emph{player}; \`e stata scelta un implementazione
a camere stagne, in modo da rendere il sistema il pi\'u modulare
possibile, così da separare (anche fisicamente, distribuendoli su varie macchine) i vari componenti ed utilizzare processi singoli per ogni istanza.\\\
\\\
Si procede quindi con la descrizione del comportamento di ogni
elemento del sistema.

\subsection{Direttore}
% XXX MELLO XXX %
Il direttore rappresenta la componente che lega ed indirizza l'intera esecuzione. Basa le proprie scelte su un database di generi e sottogeneri, descritti tramite pattern.
Ogni pattern definisce la struttura base di un ``giro'' di accordi dello specifico sottogenere, ad esempio la struttura di un classico blues bebop, indicando la sequenza di accordi e di ``intenzioni'' consigliate per ogni battuta del giro. 
Sfruttando queste informazioni ed essendo a conoscenza di ciò che è già stato suonato, il direttore decide tramite meccanismi probabilistici, battuta in battuta, le caratteristiche della prossima misura di brano da eseguire, elegge un solista per questa sezione e comunica queste informazioni ai musicisti, per poi attendere che tutti abbiano completato la propria esecuzione e ripetere il ciclo.
Ad ogni nuova battuta, utilizzando metodi probabilistici, il direttore decide inoltre se cambiare sottogenere o genere e quale scala di riferimento deve essere usata dal musicista solista, tenendo in considerazione le possibili varianti compatibili dello stesso genere.

\subsection{Musicista}
% XXX MATTE XXX %
Come il direttore, il musicista nel nostro software è un agente. 
Il suo scopo principale è quello di creare in tempo reale della musica.
Della buona musica? Ci prova, infatti il processo musicista trascorre la
sua esistenza suonando delle note che possano ``andar bene'' assieme alle note
 suonate dagli altri musicisti. Questi ultimi non vengono lasciati soli nelle 
decisioni prese durante un'improvvisazione ma il direttore li aiuta a prendere 
delle scelte che possano aver senso fra di loro e li aiuta a coordinarsi. 
Il direttore quindi, tramite un certo protocollo di comunicazione, 
invia determinati parametri globali a tutti i musicisti che a loro volta 
scandiscono il database per cercare delle note che possano avere senso nel loro 
attuale contesto. 
Ad ogni insieme di note che i musicisti ottengono in ogni passo dell'esecuzione
 è correlato un set di probabilità, il quale viene utilizzato per filtrare le 
note scelte e ad introdurre il comportamento di improvvisazione ponderata.

\subsection{Player}
% XXX FEDE XXX %
Il player è l'unico componente del progetto che non si comporta da agente 
intelligente, ma piuttosto da interprete, poiché il suo compito è suonare (in linguaggio MIDI) le note che i musicisti hanno ``scritto sui loro spartiti''.
Nel dettaglio, le informazioni che pervengono al player da ogni 
musicista sono set di note e durate (secondo la struttura descritta nella 
sezione \ref{sec:play_measure}) che compongono una battuta.
Il player resterà bloccato finché non riceverà questo dato da ognuno dei 
musicisti (il cui numero è conosciuto a priori), poi inizierà la sua esecuzione.
Essa si basa sulla scansione dei dati in input secondo una base temporale 
atomica (che corrisponde alla durata della metà di una semicroma terzinata, 
ovvero un quarantottesimo di una misura intera) con la quale si possono 
rappresentare in base numerica intera tutte le note utilizzate.
È da segnalare che non utilizziamo mai note di durata inferiore alla semicroma 
terzinata (quindi biscroma e semibiscroma, per i più avvezzi alla teoria musicale) 
per pura semplicità, poiché questo non pregiudica una buona dimostrazione del 
funzionamento del software.
\newline

Il player nasce come esecutore in tempo reale (o immediatamente successivo) 
rispetto alla creazione del brano, ma presenta una feature interessante, 
ovvero la scrittura su file midi del brano in creazione per esecuzioni future 
o studio dello spartito generato su tools come Tuxguitar\footnote{\url{http://www.tuxguitar.com.ar/}}.

Questo componente utilizza il labeling standard per gli strumenti musicali definito da General Midi\cite{generalmidi}, 
compatibile con la maggior parte dei sintetizzatori software. Abbiamo quindi scelto TiMidity++\footnote{\url{http://timidity.sourceforge.net/}} come strumento per tradurre il nostro operato in qualcosa di udibile.
