\section{Conclusioni}
In conclusione, i risultati del progetto rispecchiano le aspettative iniziali
mirate alla dimostrazione della fattibilità di un modello software in grado di improvvisare.
Attualmente, il progetto è ancora ad uno stato embrionale, ma, con le dovute
modifiche, basandosi sullo studio effettuato, potrebbe divenire un pratico
strumento didattico o una piattaforma in grado di aiutare lo sviluppo di
ricerche legate all'improvvisazione automatica.\\
\\
%TANTE COSE CHE DOVETE DIRMI VOI PERCHé NON LO SO :).
Le premesse sono state soddisfatte, ma vi sono molti problemi di fondo e certamente punti da migliorare.
L'algoritmo evoluzionistico pare essere una soluzione migliore dell'approccio stocastico, ma quest'ultimo rispecchia in modo scrupoloso le regole armoniche, restituendo così un risultato gradevole all'ascoltatore, anche se per ora slegato da uno stile preciso (dovuto anche a una scarsità di informazioni all'interno del database).
L'algoritmo evoluzionistico acquisisce invece una buona percentuale di similitudine con la melodia ideale, ma nei punti dove essi non sono simili, restituisce una sequenza di note che possono sembrare casuali.
In un certo senso, almeno per il momento, possiamo affermare che, nonostante una buona percentuale di similitudine, esso ``suona male''.\\
Con questo si ribadisce che lo stato embrionale del progetto garantisce buoni risultati sperimentali, e offre buone premesse per questa branca inesplorata.

%ma ancora molto lavoro da PERFORMARE da parte di 'sti stronzi degli sviluppatori, stronzi soprattutto perché usano sistemi operativi di merda e bevono birra invece di lavorare così l'alcol gli brucia la concentrazione e fanno lavorare il povero emarginato (un po' stronzo pure lui.).
