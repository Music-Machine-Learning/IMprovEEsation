\section{Overview dei Componenti}
Il progetto \`e composto principalmente di tre parti principali:\\
Un \textbf{direttore}, il quale svolge il compito di centralizzare l'improvvisazione,
dettando regole e decidendo i parametri generali per ogni punto dell'improvvisazione;
potrebbe in un certo senso essere visto come una sorta di \emph{coscienza comune},
la quale amministrebbe silentemente i vari musicisti, vedremo pi\'u avanti come
la natura centralizzata del direttore inoltre aiuti, ad esempio, a sincronizzare
i vari musicisti.\\
Un \textbf{player}, il quale assieme al direttore compone l'architettura centralizzata
del progetto; il compito del player \'e di ricostruire e di assemblare le varie
improvvisazioni provenienti dai vari musicisti, ha inoltre un interfaccia
modulare per salvare o riprodurre l'improvvisazione, per analogia con il direttore,
il quale \'e il punto d'ingresso del progetto, il player \'e il punto dove viene
formato l'output utile del progetto.\\
Vari \textbf{musicisti}, i quali prendono decisioni in base alla loro configurazione
e all'output del direttore, processandole secondo vari meccanismi euristici
(vedremo ad esempio implementazioni di meccanismi randomici o basati su algoritmi evolutivi).\\
\\
Al fianco di questi componenti fondamentali \'e presente un ambiente di supporto
per facilitare l'operazione, quali ad esempio il \textbf{database} dell'applicazione,
responsabile dell'immagazzinamento della conoscenza dei vari componenti, ad
esempio la rappresentazione dei vari generi e dei loro pattern collegati.
