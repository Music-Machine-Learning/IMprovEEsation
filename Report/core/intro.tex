\section{Introduzione}
L'improvvisazione musicale è sempre stata oggetto di profondo studio da parte
di musicisti e teorici musicali.\\
Radicata nella natura profondamente controversa dell'improvvisazione,
che ne fa quindi un occasione di personalizzazione di un brano da parte di un musicista,
vi è una complessità non indifferente dovuta all'impossibilità di definirla
attraverso regole e formalismi precisi.\\
La teoria musicale, nel corso della storia, ha fornito alcuni costrutti generali
a supporto delle tecniche di improvvisazione, come ad esempio
\emph{scale pentatoniche} e \emph{centri tonali}, pur rimanendo direttive lasche
in grado di lasciare al musicista diversi gradi di libertà, è quindi possibile
simulare con una certa fedeltà un improvvisazione tra più musicisti?\\
\\
Il presente progetto si pone proprio in questo senso:
si cerca, appunto, di dare una risposta alla precedente domanda, costruendo un
improvvisazione automatica tra più musicisti virtuali; ovviamente ciò non affatto
banale, come specificato in precedenza, analizzeremo quindi le problematiche
e alcuni metodi di risoluzione, costruendo quindi un \emph{proof of concept} finalizzato
ad evidenziare sia la fattibilità, sia le difficoltà che uno scenario di questo
genere può presentare.\\
Come sarà approfondito nella sezione successiva, uno scenario di questo tipo
non è mai stato preso in considerazione, ma probabilmente potrebbe essere
un buon punto di partenza per la costruzione di una piattaforma a supporto della
didattica oppure alla realizzazione di macchine in grado di mescolarsi e integrarsi
in una sessione di improvvisazione tra musicisti reali
\footnote{Esistono già macchine in grado di suonare strumenti reali seguendo
delle basi predefinite, citiamo ad esempio i famosi \emph{compressorhead}}.
