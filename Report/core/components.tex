\section{Componenti del Sistema}
% XXX BERA XXX %
Come indicato in precedenza, i principali componenti in stretto contatto
tra loro sono delle seguenti tre categorie, \'e stata scelta un implementazione
non a camere stagne tra di essi, in modo da rendere il sistema il pi\'u modulare
possibile, potendo separare (anche fisicamente, distribuendoli su varie macchine)
i vari componenti, utilizzando dei processi singoli per ogni istanza del singolo.\\\
\\\
Si procede quindi con la descrizione dettagliata del comportamento di ogni
elemento del sistema.

\subsection{Direttore}
% XXX MELLO XXX %

\subsection{Musicista}
% XXX MATTE XXX %
Come il direttore, il musicista nel nostro software è essenzialmente un processo. Il suo scopo principale è quello di creare in tempo reale della musica. Della buona musica? Ci prova; infatti il processo musicista trascorre la sua esistenza suonando delle note che possano "andar bene" assieme alle note suonate dagli altri musicisti. Questi ultimi non vengono lasciati soli nelle decisioni prese durante un'improvvisazione ma il direttore li aiuta a prendere delle scelte che possano aver senso fra di loro e li aiuta a coordinarsi. Il direttore quindi, tramite un certo protocollo di comunicazione, invia determinati parametri globali a tutti i musicisti che a loro volta scandiscono il database per cercare delle note che possano avere senso nel loro attuale contesto. Ad ogni insieme di note che i musicisti ottengono ad ogni passo dell'esecuzione è correlato un set di probabilità, il quale viene utilizzato per filtrare le note scelte da utilizzare e ad introdurre il comportamento di improvvisazione.

\subsection{Player}
% XXX FEDE XXX %
