\section{Ragionamento Automatico}
Quando un musicista o un gruppo di musicisti eseguono un
improvvisazione, sono molteplici i fattori che incidino nelle loro azioni. 
Tra questi sicuramente hanno molta influenza le emozioni
individuali e reciproche dei musicisti, oppure il "feeling" che c'è tra
loro.\\
Indubbiamente anche se non è l'unico fattore in gioco, 
non può mancare un certo tipo di ragionamento
nella scelte effettuate da ogni musicista, che sa in qualche modo come
agire per produrre un risultato che sia apprezzabile per se stesso e per gli
altri. Per un software che genera delle improvvisazioni musicali, il fattore 
emotivo non può però essere escluso. Anche se le macchine non sono in grado
di provare delle emozioni reali, possono sempre simulare dei
comportamenti che sembrino essere influenzati da delle emozioni reali, e
forse possono anche scaturire emozioni a chi interagisce con tali
macchine. Quest'ultimo è un obbiettivo arduo ma sicuramente interessante
per software di questo tipo. I puristi della musica e dell'arte
potrebbero sollevare una critica dicendo che un musicista che non prova
emozioni non è in grado di sollevare emozioni verso chi li ascolta.
Lo studio emotivo nell'intelligenza artificiale è un campo aperto e
osservato con interesse dal mondo scientifico, risulta quindi stimolante
proporne un applicazione di studio.\\
In IMprovEEsation il ragionamento dei musicisti è fortemente influenzato
dal ragionamento del direttore che potrebbe essere visto come una sorta
di "feeling" o di sintonia tra i vari musicisti improvvisatori, che in
un caso reale non avrebbero un coordinatore centrale.\\
In questa sezione vedremo meglio come ragionano il direttore e il
musicista durante una sessione di improvvissazione.
\label{thinking}
% XXX MELLO XXX %
\subsection{Mente del Direttore}
% XXX MELLO XXX %
Il compito del direttore è quello di controllare il flusso dell'improvvisazione nella sua globalità, istruendo i musicisti su cosa e come ``sarebbe corretto'' suonare. 
Per ogni battuta, la mente del gruppo stabilisce in una serie di step successivi:
\begin{itemize}
\item il genere (e sottogenere/variante)
\item gli accordi di riferimento per l'accompagnamento
\item i centri tonali per l'improvvisazione
\item il musicista solista
\end{itemize}

\noindent
Inoltre vengono settati in maniera statica (ma è prevista la possibilità di deciderli dinamicamente):
\begin{itemize}
\item velocità
\item tempo
\item dinamiche ed intenzione
\item priorità degli argomenti delle query per il musicista (vedi \ref{musicista:priorargs}) % FIXME ref %
\end{itemize}

Questi valori sono letti dal file di configurazione e dal database, nello specifico i pattern dei generi dichiarano dinamiche ed intenzione per ogni battuta ed un set di priorità di argomenti da scartare.

Per lasciare massima libertà di sperimentazione, tutte le soglie variabili degli algoritmi probabilistici sono lette dal file di configurazione durante la fase di inizializzazione del direttore.
Un file di configurazione di esempio è fornito insieme al programma, i valori qui contenuti sono quelli discussi nei prossimi paragrafi e rappresentano ciò che si è ritenuti più ``sensato'', sfruttando anche la consulenza di musicisti esperti. % FIXME non vorrei averla cagata fuori.. %

\subsubsection{Scelta del genere}
Per prima cosa è necessario stabilire, per ogni battuta, se è possibile cambiare l'intento globale dell'improvvisazione, la maniera di ottenere questo cambiamento drastico è scegliere un genere o sottogenere differente da quello attualmente in uso.

Il concetto di sottogenere identifica una serie di varianti dello stesso genere musicale, non necessariamente intercambiabili l'una con l'altra, ma ceramente appartenenti alla stessa macrocategoria, per esempio sono stati definiti pattern per blues base, che è il classico blues in dodici battute, il blues bebop ed il blues Coltrane che si basa sulla famosa progressione blues inventata dal noto sassofonista.

Prima di tutto vanno considerate due situazioni differenti, in base alla battuta che si andrà a creare:
\begin{enumerate}
\item la battuta è la prima di un ``giro'';
\item la battuta è una battuta qualunque all'interno di un ``giro'' (e.g. non la prima).
\end{enumerate}

Il comportamento implementato è quello di NON cambiare mai genere se non sulla prima battuta di un giro, in modo da caratterizzare ogni parte dell'improvvisazione in maniera sufficientemente chiara e plausibile. 

In ultimo viene deciso, attraverso una soglia percentuale, se è il caso di cambiare sottogenere oppure passare proprio ad un genere differente.
Poiché il dataset creato non contiene altri generi propriamente implementati all'infuori del blues, questa soglia è fissa sul cambio di sottogenere; in ogni modo un valore plausibile è il 75\% di probabilità di passare ad un sottogenere, invece di cambiare il genere in toto.

\subsubsection{Accordi di riferimento}
Una volta stabilito il genere è necessario decidere gli accordi cha andranno suonati dai musicisti di accompagnamento.
Il primo riferimento considerato è quello dichiarato dal pattern del genere: ogni pattern specifica la progressione di accordi tipica di un giro di quello specifico sottogenere (ad esempio il blues base sarà descritto dalla sequenza di accordi maggiori con la settima minore costruiti sui gradi \texttt{I-I-I-I-IV-IV-I-I-V-V-I-I}).
% FIXME forse qui basta un riferimento alla spiegazione dei pattern %
Basandosi su questa informazione ed essendo a conoscenza di quale battuta si sta per suonare all'interno della progressione, il direttore stabilisce se è plausibile cambiare l'accordo di riferimento rispetto a quello dichiarato dal pattern: si considerano accordi sostituibili quelli che coprono l'intera battuta oppure sono una cadenza\footnote{La cadenza è la successione all'interno della stessa battuta di un accordo minore di settima minore sul secondo grado e un accordo maggiore di settima minore sul quinto grado; questa circstanza è armonicamente assimilabile ad un accordo sul primo grado}.

Il secondo nodo della decisione è guidato dalla posizione della battuta all'interno del ``giro'', analogamente a quanto accade per la sostituzione del genere:
\begin{itemize}
\item se la battuta è la prima di un giro, l'accoro non dovrebbe essere mai sostituito (tendenzialmente la prima battuta è quella che dichiara la tonalità)
\item se la battuta non è la prima, con bassa probabilità si potrà cambiare l'accordo.
\end{itemize}



\subsubsection{Centri tonali}

\subsubsection{Scelta del solista}


\subsection{Mente del Musicista}
% XXX MELLO XXX %
Come ragione il musicista quando decide le note?
