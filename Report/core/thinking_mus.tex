\subsection{Mente del Musicista}
%   \begin{algorithm}
%   \caption{Musician improvisation cycle algorithm}\label{algo-musician}
%   \begin{algorithmic}[1]
%   \Function{musician\_main}{}
%   	\State $id \gets$send\_subscription($director$)
%   	\State send\_subscription($player$)
%   	
%   	\While{\textbf{not} EOI}\Comment{Improvise until a End Of Exception is raised}
%   	\State $measure\_info \gets$ recv($director$)
%   	\State $measure\_to\_play \gets$ compose\_measure($measure\_info$)
%   	\State send($player$, $measure\_to\_play$);
%   	\EndWhile
%   \EndFunction
%   \end{algorithmic}
%   \end{algorithm}
%
%   % TODO spiegare il primo algoritmo %
%
%   \begin{algorithm}
%   \caption{Musician improvisation cycle algorithm in genetic mode}\label{algo-musician-get}
%   \begin{algorithmic}[1]
%   \Function{musician\_genetic\_mainz}{}
%   	\State $id \gets$send\_subscription($director$)
%   	\State send\_subscription($player$)
%   	
%   	\While{$measure\_info_{bpm} \not= BPM\_EOI$}\Comment{Improvise until a BPM EOI is
%   received}
%   	\State $measure\_info \gets$ recv($director$)
%   	\State $measure\_to\_play \gets$ compose\_measure($measure\_info$)
%   	\State store\_measure(measure\_info, measure\_to\_play)
%   	\State ack($director$)
%   	\EndWhile
%   	\State genetic\_loop()	
%   	\State ack($director$)
%   	\State Perform the MUSIC\_MAIN function loop with the new generated measures
%   \EndFunction
%   \end{algorithmic}
%   \end{algorithm}
%

Come già introdotto nelle sezioni precedenti, il compito principale dei musicista è
quello di decidere le note da suonare basondosi sulle informazioni
ricevute dal direttore che le manda in broadcast in rete per ogni misura
da improvvisare. L'algoritmo \ref{algo-musician-measure} mostra come un
musicista compone una misura da improvvisare basandosi sulle
informazioni ricevute dal direttore su di essa, $data_{MSR}$, e sulle
informazioni inerenti alla sua natura, $data_{MCN}$.  

\begin{algorithm}
\caption{Musician measure improvisation algorithm}
\label{algo-musician-measure}
\begin{algorithmic}[1]
\Function{composeMeasure}{$data_{MSR}$, $data_{MCN}$}
\For{$q\gets 0$ \textbf{to} $nq$} \Comment{$nq$, the number of
quarters in a measure, is time signature dependent}
	\State $n,i\gets0$
	\While{$n = 0$}
		\State $(n,quarters)\gets$ searchQuarters($db$, $data_{MSR}$, $data_{MCN}$, $prioargs$, $i$++)
	\EndWhile
	\State $quarter\gets$ random($quarters$)
	\State $(n,sqs)\gets$ getSemiquavers($db$, $quarter$)
	\State $s\gets0$
	\For{$i\gets 0$ \textbf{to} 4} \Comment{For the all quarter
subdivistions}
		\State $sq\gets sqs[s]$
		\If {$s=n$ \textbf{or} $sq_{pos} \neq i$}
			\State extendPrevNote()
			\State \textbf{continue}
		\EndIf
		\If {$sq_{pc}<rand()$}\Comment{Probabilty of note change}
			\State extendPrevNote()
		\Else
			\State decideNote($sq_{pnote}$)
		\EndIf
		\State $s$++
		
	\EndFor
\EndFor
\EndFunction
\end{algorithmic}
\end{algorithm}

All'inizio Il musicista cerca nel DB dei \emph{quarter} che possano
avere un riscontro con i due set di informazioni. Se non trova nessun
quarto che fa "match" con tutte le informazioni ripete la ricerca scartando
gli argomenti con priorità più bassa, fino a quando non ne trova almeno
uno. Tra i quarti trovati, se più di 1, ne viene scelto uno random
\footnote{In una futura estensione, un quatro potrebbe venir scelto con
una politica migliore, ad esempio facendo un confronto ulteriore tra quelli
scelti e i precedenti.} di cui il processo ottiene tutte i record
\emph{semiquaver} associate.

%TODO decide note

