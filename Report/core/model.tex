\section{Modello del Dominio}
% XXX FEDE XXX %
%TODO i componenti alla larga (overview)

Questa sezione descriverà la composizione del nostro progetto alla luce di ciò che è gia stato fatto e ciò che vogliamo introdurre.
Il progetto ha come priorità l'esecuzione di un brano imrovvisato da parte di più musicisti, i quali non hanno (almeno per il momento) alcuna coscienza della presenza di altri musicisti.
Come mostrato in figura ~\ref{fig:dom}, la struttura del sistema consta di alcuni componenti differenti, che descriveremo nelle sezioni seguenti, organizzati esattamente come in un'orchestra reale (fatta eccezione per il player):
il direttore d'orcherstra trasferisce le proprie informazioni rgenerali riguardo all'esecuzione a un numero arbitrario di musicisti, i quali eseguono una parte definita del brano e la passano a un player, che si occupa di tradurre in contemporanea ogni parte ``scritta'' da un musicista appunto in musica.
È quasi come se i musicisti in questo caso, invece di produrre il suono loro stessi, producessero lo spartito misura per misura.

\begin{figure}[H]
\centering
\includegraphics[scale=0.30]{img/model.png}
\caption{Schema dello scenario di progetto}
\label{fig:dom}
\end{figure}

Nel nostro caso, sia il direttore sia i musicisti sono da considerarsi agenti intelligenti, mentre il player è un mero esecutore (come vedremo nelle descrizioni dettagliate nei prossimi capitoli).

Tali agenti intelligenti operano in un ambiente che potremmo descrivere con la notazione PEAS:
\begin{itemize}
 \item Parzialmente osservabile, poichè i musicisti conoscono solo lo stato corrente del direttore, ovvero soltanto ciò che esso decide di far suonare più, ovviamente, il proprio stato, ma non conosce ciò che gli altri musicisti eseguono.
 Inoltre, il direttore non conosce lo stato interno dei musicisti, ma si limita a decidere i parametri della misura seguente mediante un proprio algoritmo interno.
 Questo aspetto è presente in parte per semplificare la struttura attuale, ma potrebbe essere in seguito modificato.
 \item Strategico, poichè lo stato successivo dell'ambiente non è determinato dalle mosse di un agente, ma deve tener conto anche delle mosse degli altri agenti, che, pur essendo cooperativi, potrebbero risultare imprevedibili.
 Il direttore decide le proprie mosse che influenzano la globalità del sistema, ma non ha controllo su ogni singolo agente.
 \item Episodico nel caso del musicista, poichè le azioni intraprese da quest ultimo non hanno ripercussioni future nè costituiscono un parametro di decisione nell'episodio seguente.
 La percezione del musicista è definita dalle informazioni pervenute dal direttore.
 Nel caso del direttore invece l'ambiente assume una connotazione sequenziale, poichè alcuni parametri dell'azione corrente determinano una probabilità di passaggio a differenti azioni successive possibili.
 \item Statico, poichè solo gli agenti coinvolti possono variare l'ambiente.
 \item Discreto poichè, pur operando in un'ottica real-time, le azioni degli agenti si basano su unità di tempo atomiche uguali per tutti, come, del resto, la teoria musicale impone.
 \item Multiagente, anche se le interazioni reali fra agenti sono relativamente scarse. Questo fa di un ambiente concettualmente cooperativo, in realtà un ambiente composto da unità che dagli altri agenti possono essere viste come stocastiche e imprevedibili.
 \end{itemize}

%https://www.lucidchart.com/documents/edit/67c03865-48ea-4f6a-b052-e9f9c4cd8196?
