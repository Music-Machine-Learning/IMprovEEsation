\section{Stato dell'Arte}
% XXX FEDE XXX %
Questa sezione tratta del lavoro che è già stato compiuto nell'ambito della composizione e dell'improvvisazione della macchina.
È da precisare che questo progetto non si è inizialmente fondato su un'idea esistente, poichè siamo voluti partire da un'idea che è scaturita da noi e l'abbiamo voluta sviluppare.
Le pubblicazoni presenti in questa sezione sono da considerarsi spunti.
\newline

Una delle applicazioni più utilizzate da ciò che è chiamato ``Machine Composition'' è la possibilità di creare una vera e propria band di backup con la quale un improvvisatore può suonare.
Perciò quello che interessa la maggior parte di questi software è la creazione di una melodia standard, non articolata, che segua una mappa di accordi fornita dall'utente e un certo pool di stili preimpostati (che possono essere molto numerosi), con l'obiettivo di dare all'utente la facoltà di suonare il proprio strumento con una band che esegue il background.
L'improvvisazione è sempre il risultato finale, ma non è esattamente quello che vogliamo ottenere.
Fra questi software citiamo Band-In-A-Box\cite{biab}, che crea basi di accompagnamento in diverse divisioni, in tutte le possibili tonalità e accordi.
Recentemente è in grado di esibirsi in assoli di chitarra e sassofono, che però si riferiscono sempre a standard compresi nel database dei pattern del programma.
Questo software ottiene ottimi risultati nell'accompagnamento, ma non concede piena libertà ai suoi musicisti digitali, ovvero non considera che ``tutto potrebbe accadere'' come in un'improvisazione reale.
\\
Un altro aspetto molto affine con il nostro lavoro è una branca dell'IA chiamata Evolutionary Music, che è divenuta una vera e propria materia di studio a sè stante\cite{evomus}.
Sono stati effettuati molti tentativi di integrare l'improvvisazione e la composizione con gli algoritmi genetici, due esempi possono essere GenDash\cite{gendash}, che aiuta il musicista a comporre  musica, e GenJam\cite{genjam}, il quale output è una jam session, che utilizza i risultati di Band-In-A-Box per poi mutarli a seconda del feeling che il brano sta assumendo, ad esempio, se il solista tende ad accelerare o a inserire un numero di note maggiore, anche la batteria e il basso addenseranno più colpi rendendo il ritmo più incalzante pur essendo la base la stessa.
\\
L'Evolutionary Music si incrocia con la Computer Music in un ambito detto Machine Improvisation, che utilizza diversi sistemi per simulare l'improvvisazione di un solista su una base musicale parzialmente nota.
Esistono i metodi statistici, basati su Catene di markov, HMM e processi stocastici\cite{hmm}.
Questi hanno dato origine a progetti come il Continuator\cite{cont}, il quale fa uso di modellazioni di stili non-real time\cite{dubnov}.
\newline

È da precisare che un programma in grado di produrre un brano completamente improvvisato non è mai stato realizzato, ma si è sempre cercato di partire da una base musicale definita (facendo improvvisare un solo solista) oppure musicisti virtuali che imparano da un musicista fisico in real time per poterlo accompagnare.
L'intento del presente progetto è invece quello di creare ciò che pù si avvicina a una Jam session, ovvero strumentisti che si ritrovano in una sala prove e seguono una serie di direttive generiche per produrre un brano senza una base esistente.
