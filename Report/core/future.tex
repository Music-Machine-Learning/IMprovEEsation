\section{Sviluppi Futuri}
% XXX TUTTI ASSIEME XXX %
Come in ogni software, ci sono diversi aspetti che
potrebbero essere estesi e migliorati. Tali estensioni e miglioramenti,
discussi di seguito, sono decisamente facilitati dall'alta modularità
dei componenti.
\subsection{Miglioramenti del Direttore}
% XXX MELLO XXX %
Come già dichiarato al paragrafo \ref{dir-tonalZones}, un primo miglioramento che richiederebbe poco sforzo è l'implementazione delle varianti di improvvisazione durante la scelta della scala di riferimento per i solisti. Questa miglioria non è stata ancora completamente implementata perché richiederebbe un refactoring di parte del codice decisionale del direttore, ma non si tratta di una ristrutturazione eccessivamente onerosa, dato che il resto del sistema prevede già la presenza di questa feature.

Un altro passo avanti potrebbe essere la modifica dinamica dei parametri attualmente statici di ogni battuta generata. Questo è un argomento delicato, poiché la maggior parte di questi parametri hanno carattere estremamente istintivo, non esistono regole che definiscano come modificarli in maniera corretta o sbagliata.
La maniera più sensata di affrontare un simile problema potrebbe essere quella di implementare un meccanismo di machine learning basato su training manuale che sfrutti l'opinione di un numero sostanzioso di ascoltatori eterogenei e valuti di conseguenza quali modifiche sono più o meno apprezzabili.
% FIXME trainig manuale non credo sia corretto.. %

Simili meccanismi di machine learning potrebbero portare anche a risultati interessanti per la scelta di tutti gli altri fattori determinati in maniera probabilistica, nonché una possibile evoluzione delle strutture dati contenute nel database, aggiornando i pattern con le modifiche introdotte da scelte pseudocasuali di accordi e le relazioni tra sottogeneri compatibili.

\subsection{Sviluppo della Conoscenza}
% XXX FEDE XXX %
% Spieghiamo qui o in sviluppi futuri? Comnque potremmo proporre nel futuro di salvare nel db 
% il risultato del genetico "facendo un match" dei quarter che sono usciti dal genetico con quelli che già
% ci sono nel db "aggiustando" le probabilità che già ci sono. Quelle che non ci sono possiamo aggiungerle. 

% FIXME forse andrebbe detto qui dentro che sarebbe il caso di rendere genetico in qualche modo anche il direttore %

Meritano una menzione a parte gli sviluppi futuri dedicati alla parte di machine learning.
In questo momento, l'algoritmo evoluzionistico fa in modo che un musicista ``impari'' a improvvisare data una sequenza ideale come traccia.
Come rendere questo apprendimento permanente?\\
La nostra proposta per un futuro sviluppo è incrociare l'algoritmo evoluzionistico con l'approccio sulle probabilità.
Idealmente, un utente darebbe in pasto all'algoritmo evoluzionistico il musicista con il brano ideale finché ciò che viene prodotto non sarà buono.
Avvenuto questo, il prodotto dell'algoritmo dovrebbe venire trasformato in un set di probabilità (meglio ancora se si considerano più output di tale algoritmo, per facilitare la trasformazione in percentuali) che potrà aggiornare il database dei pattern, ovvero le probabilità già presenti sul database alle quali si fa riferimento durante l'approccio probabilistico.
\newline

Una seconda miglioria, affine con le tecniche esistenti di machine composition e machine improvisation, sarebbe quella di considerare gli altri musicisti e le loro esecuzioni.
Ciò che l'algoritmo evoluzionistico potrebbe fare in questo caso è l'alterazione della funzione che computa la similitudine in modo che vengano introdotte grandezze che non confrontano più il pool genetico con il pattern ideale, ma anche con le scelte degli altri strumentisti.
\newline

Abbiamo notato che l'algoritmo evoluzionistico tende a girare attorno a un punto fisso, dopo un repentino miglioramento iniziale, dipendente dalla lunghezza del brano.
Una terza miglioria consiste nel migliorarlo per aumentarne le prestazioni, soprattutto su brani di lunghezza consistente, per poter spingere quel punto fisso a percentuali di similitudine più alta.
