\section{Ragionamento Automatico}
Quando un musicista o un gruppo di musicisti eseguono un
improvvisazione, sono molteplici i fattori che incidono sulle loro azioni. 
Tra questi sicuramente hanno molta influenza le emozioni
individuali e reciproche dei musicisti, oppure il "feeling" che c'è tra
di loro.\\
Per un software che genera delle improvvisazioni musicali, il fattore 
emotivo non può essere escluso.
Anche se le macchine non sono in grado
di provare delle emozioni reali, possono sempre simulare dei
comportamenti che sembrino, ai sensi di un osservatore esterno, 
essere influenzati da esse, e forse possono anche scaturire emozioni a chi 
interagisce con tali
macchine. Quest'ultimo è un obiettivo arduo ma sicuramente interessante
per software di questo genere. I puristi della musica e dell'arte
potrebbero sollevare una critica dicendo che un musicista che non prova
emozioni non è in grado di sollevare emozioni verso chi li ascolta.
Lo studio emotivo nell'intelligenza artificiale è un campo aperto e
osservato con interesse dal mondo scientifico, risulta quindi stimolante
proporne un applicazione di studio.\\

Inoltre, anche se non è l'unico fattore in gioco, 
non può mancare un certo tipo di ragionamento
nella scelte effettuate da ogni musicista, che sa in qualche modo come
agire per produrre un risultato che sia apprezzabile per se stesso e per gli
altri.\\
In \emph{IMprovEEsation} il ragionamento dei musicisti è fortemente influenzato
dal ragionamento del direttore che potrebbe essere visto come una sorta
di sintonia tra i vari musicisti improvvisatori, che in
un caso reale non avrebbero un coordinatore centrale.\\
In questa sezione vedremo meglio come ragionano il direttore e i
musicisti durante una sessione di improvvisazione.
\label{thinking}
% XXX MELLO XXX %
\subsection{Mente del Direttore}
Come ragiona il direttore quando decide i pattern?
\subsection{Mente del Musicista}
\begin{algorithm}
\caption{Musician improvisation cycle algorithm}\label{algo-musician}
\begin{algorithmic}[1]
\Function{musician\_main}{}
	\State $id \gets$send\_subscription($director$)
	\State send\_subscription($player$)
	
	\While{\textbf{not} EOI}\Comment{Improvise until a End Of Exception is raised}
	\State $measure\_info \gets$ recv($director$)
	\State $measure\_to\_play \gets$ compose\_measure($measure\_info$)
	\State send($player$, $measure\_to\_play$);
	\EndWhile
\EndFunction
\end{algorithmic}
\end{algorithm}

% TODO spiegare il primo algoritmo %

\begin{algorithm}
\caption{Musician improvisation cycle algorithm in genetic mode}\label{algo-musician-get}
\begin{algorithmic}[1]
\Function{musician\_genetic\_main}{}
	\State $id \gets$send\_subscription($director$)
	\State send\_subscription($player$)
	
	\While{$measure\_info_{bpm} \not= BPM\_EOI$}\Comment{Improvise until a BPM EOI is
received}
	\State $measure\_info \gets$ recv($director$)
	\State $measure\_to\_play \gets$ compose\_measure($measure\_info$)
	\State store\_measure(measure\_info, measure\_to\_play)
	\State ack($director$)
	\EndWhile
	\State genetic\_loop()	
	\State ack($director$)
	\State Perform the MUSIC\_MAIN function loop with the new generated measures
\EndFunction
\end{algorithmic}
\end{algorithm}

