\subsection{Mente del Musicista}
%   XXX: DO NOT DELETE THE COMMENTED ALGORITHMS, MAY BE USEFUL FOR
%   FUTURE DOCUMENTATION!!
%   \begin{algorithm}
%   \caption{Musician improvisation cycle algorithm}\label{algo-musician}
%   \begin{algorithmic}[1]
%   \Function{musician\_main}{}
%   	\State $id \gets$send\_subscription($director$)
%   	\State send\_subscription($player$)
%   	
%   	\While{\textbf{not} EOI}\Comment{Improvise until a End Of Exception is raised}
%   	\State $measure\_info \gets$ recv($director$)
%   	\State $measure\_to\_play \gets$ compose\_measure($measure\_info$)
%   	\State send($player$, $measure\_to\_play$);
%   	\EndWhile
%   \EndFunction
%   \end{algorithmic}
%   \end{algorithm}
%
%
%   \begin{algorithm}
%   \caption{Musician improvisation cycle algorithm in genetic mode}\label{algo-musician-get}
%   \begin{algorithmic}[1]
%   \Function{musician\_genetic\_main}{}
%   	\State $id \gets$send\_subscription($director$)
%   	\State send\_subscription($player$)
%   	
%   	\While{$measure\_info_{bpm} \not= BPM\_EOI$}\Comment{Improvise until a BPM EOI is
%   received}
%   	\State $measure\_info \gets$ recv($director$)
%   	\State $measure\_to\_play \gets$ compose\_measure($measure\_info$)
%   	\State store\_measure(measure\_info, measure\_to\_play)
%   	\State ack($director$)
%   	\EndWhile
%   	\State genetic\_loop()	
%   	\State ack($director$)
%   	\State Perform the MUSIC\_MAIN function loop with the new generated measures
%   \EndFunction
%   \end{algorithmic}
%   \end{algorithm}
%
\label{sec:musician_think}
Come già introdotto nelle sezioni precedenti, il compito principale dei musicista è
quello di decidere le note da suonare basandosi sulle informazioni
ricevute dal direttore che le manda in broadcast in rete per ogni misura
da improvvisare. L'algoritmo \ref{algo-musician-measure} mostra come un
musicista compone una misura da improvvisare basandosi sulle
informazioni ricevute dal direttore su di essa, $data_{MSR}$, e sulle
informazioni inerenti alla sua natura, $data_{MCN}$. 

\begin{algorithm}[H]
\caption{Musician measure improvisation algorithm}
\label{algo-musician-measure}
\begin{algorithmic}[1]
\Function{improviseMeasure}{$data_{MSR}$, $data_{MCN}$}
	\For{$q\gets 0$ \textbf{to} $nq$} \Comment{$nq$, the number of
	quarters in a measure, is time signature dependent}
		\State $n,i\gets0$
		\While{$n = 0$}
			\State $(n,qts)\gets$ searchQuarters($db$, $data_{MSR}$, $data_{MCN}$, $prioargs$, $i$++)
		\EndWhile
		\State $qt\gets$ random($qts$)
		\State $(n,sqs)\gets$ getSemiquavers($db$, $qt$)
		\State $s\gets0$
		\For{$i\gets 0$ \textbf{to} 4} \Comment{For the all quarter
	subdivistions}
			\State $sq\gets sqs[s]$
			\If {$sq_{pos} \neq i$}
				\State extendPrevNote()
				\State \textbf{continue}
			\EndIf
			\If {$sq_{pc}<rand()$}\Comment{Probabilty of note change}
				\State extendPrevNote()
			\Else
				\State $note \gets$decideNote($sq_{pnotes}$)
				\State storeNote(midiTransform($note$))
			\EndIf
			\State $s$++
			
		\EndFor
	\EndFor
\EndFunction
\end{algorithmic}
\end{algorithm}

Inizialmente il musicista cerca nel DB, quei \emph{quarter} (da adesso
$qt$) che possano
avere un riscontro con i due set di informazioni presi in input dalla
funzione. Complessivamente gli argomenti che influiscono nella ricerca
sono:
\begin{enumerate}
\item Posizione del quarto nella misura
\item Fondamentale dell'accordo
\item Modo dell'accordo
\item Genere
\item Dinamica
\item Mood
\item Scala
\item Famiglia dello strumento\footnote{Uno tra i 16 raggruppamenti
di strumenti come definito dal General
Midi} %TODO biblio ref http://www.midi.org/techspecs/gm1sound.php#instrument
\item Modalità solista. Viene abilitata da un musicista se si propone come
possibile solista. Se ci sono più solisti nella band, il musicista deve
controllare se è il suo turno da solista prima di poter abilitare questa
modalità.
\end{enumerate}
Nello specifico il primo argomento viene semplicemente calcolato tramite un
contatore progressivo; gli argomenti dal secondo al settimo vengono
specificati dal direttore; l'ottavo fa parte della natura del musicista
e viene selezionato al suo avvio; infine il nono è dipendente sia dalla
natura del musicista sia dalle informazioni ricevute dal direttore.\\
Se il musicista non trova nessun quarto che fa "match" con tutte le
informazioni, ripete la ricerca scartando
gli argomenti con priorità più bassa, fino a quando non ne trova almeno
uno. Le priorità sono definite come valori all'interno di un vettore
$prioargs$ di dimensione 9. Per ogni elemento del vettore corrisponde un valore di priorità
associato ad uno degli argomenti sopra elencati.\\
Come già accennato nel paragrafo \ref{database}, il vettore prioargs può
essere inviato dal direttore in quanto associato al pattern da lui
scelto, oppure può venire richiesto dal musicista
stesso se per qualche motivo\footnote{Ad esempio nel caso di un
musicista batterista o solista già esposti nel paragrafo \ref{database}.} 
non vuole seguire le priorità legate al pattern.\\
Tra i quarti trovati, se più di 1, ne viene scelto uno random
\footnote{In una futura estensione, un quarto potrebbe venir scelto con
una politica migliore, ad esempio facendo un confronto ulteriore tra quelli
scelti e i precedenti.} di cui il processo scarica dal DataBase tutti i record
\emph{semiquaver} (da adesso $sq$) associati.\\
È importante precisare che ad ogni $qt$ possono essere
associate un numero minore o uguale di 4 $sq$. In questo modo si
riescono a rappresentare delle note di durata superiore ad $1\over{16}$ 
semplicemente estendendo la durata di una nota se nei suoi sedicesimi successivi 
non sono presenti record $sq$ associati al quarto
corrispondente.\\
Di fatti nell'algoritmo preso in analisi, subito dopo l'ottenimento
delle $sq$ associate al $qt$ corrente si controllano tutte le
suddivisioni di quest'ultimo. 
Se la posizione nel quarto della $sq$ 
di indice $s$ non corrisponde all'indice $i$, vuol dire che per quella
suddivisione $i$ non è associata alcuna $sq$ e pertanto
viene estesa quella precedente senza effettuare ulteriori controlli.
In caso contrario si procede a verificare se la probabilità di cambio
nota $pc$ 
\footnote{Tutti i valori di probabilità sono normalizzati
nell'intervallo [0,1]} relativo alla semicroma $s$ è più alto di valore 
generato random\footnote{Tutti i valori generati random sono presi con
distribuzione di probabilità uniforme, se non diversamente specificato}. 
Se la condizione non viene verificata allora viene
estesa la durata della nota precedente ma viene anche incrementato il
contatore $s$ in quanto la $sq$ corrente è stata considerata e scartata.
Se invece la $pc$ supera la barriera del random, si procede con la
funzione di decisione di quale nota scegliere per la semicroma corrente,
meccanismo esposto nell'algoritmo \ref{algo-musician-note}.

\begin{algorithm}[H]
\caption{Musician note decision algorithm}
\label{algo-musician-note}
\begin{algorithmic}[1]
\Function{decideNote}{$pnotes$}
	\State \textbf{let} $pref$ an array of [0,1] normalized values
	\State $pref[0]\gets pnotes[0]$
	\For {$i \gets 1$ \textbf{to} $13$} \Comment{12 Notes plus the rest}
		\State $pref[i] \gets pref[i - 1] + pnotes[i]$
	\EndFor
	\State $r \gets random()$
	\State $i \gets 0 $
	\While{$r > pref[i]$}
		\State $i$++	
	\EndWhile
	\State \textbf{return} $i$+1 \Comment{$i$ is the index of the
target previous value}
\EndFunction
\end{algorithmic}
\end{algorithm}

La funzione \emph{decideNote} prende in input il vettore di probabilità
$pnotes$ definito come nell'equazione \ref{eq-pnotes} da cui costruisce
un vettore di somme prefisse $pref$. Viene infine lanciato un numero
random $r$ e si rimane nel ciclo in $pref$ finché $r$ non è minore o uguale al
valore corrente di $pref$, la cui posizione in tal caso, indica la nota
da scegliere.\\
Questo metodo fornisce una soluzione in grado di selezionare più
preferibilmente le note con maggiore probabilità di essere scelte
rispetto le altre. L'esempio seguente può essere d'aiuto per comprendere
meglio l'applicazione dell'algoritmo, per semplicità si assume che la
lunghezza dei vettori sia pari a 7.
\begin{align*}
pnotes_1 = [0.00,\underline{0.50},0.00,0.05,0.25,0.20,0.00]\\
pref_1   = [0.00,0.50,0.50,0.55,0.80,1.00,1.00]\\\\
pnotes_2 = [0.00,0.05,0.10,0.00,0.10,0.00,\underline{0.75}]\\
pref_2   = [0.00,0.05,0.15,0.15,0.25,0.25,1.00]\\\\
pnotes_3 = [0.25,0.00,0.00,0.25,0.25,0.00,0.25]\\
pref_3   = [0.25,0.25,0.25,0.50,0.75,0.75,1.00]\\\\
\end{align*}
\begin{center}
  \begin{tabular}{ | c | c | c | c | c | c | }
	\hline
             & $r1 = 0.15$  & $r2 = 0.33$ & $r3 = 0.50$ & $r4 = 0.68$ &
$r5 = 0.88$ \\ \hline
    $pnotes1$  & 1 & 1 & 1 & 4 & 5 \\ \hline
    $pnotes2$  & 2 & 6 & 6 & 6 & 6 \\ \hline
    $pnotes3$  & 0 & 3 & 3 & 4 & 5 \\
	\hline
  \end{tabular}
\end{center}

Nella tabella dell'esempio ad ogni $pnotes$ ed $r$ corrisponde un indice
dei vettori di probabilità rispettivi. Nell'esempio vengono mostrati dei
valori random arbitrari la cui media è però significativa in quanto
tende a $0.5$ come per una successione di valori random nell'implementazione 
reale in quanto la distribuzione di probabilità che genera i numeri
random è uniforme. Anche se l'esempio è restrittivo, si può notare che
per $pnotes_1$ e $pnotes_2$, gli indici scelti maggiormente sono quelli 
con probabilità di scelta più alta e viceversa, mentre per $pnotes_3$, si 
può notare che vengono scelti circa lo stesso numero di volte 
tutti gli indici con probabilità non zero in quanto tra loro
equiprobabili.\\\\
Oltre alla funzione di decisione di nota singola, il musicista
predispone anche della funzione di decisione di un accordo. Quest'ultimo
metodo viene utilizzato solo se indicato nella natura del musicista in
questione tramite il parametro $chord\_mode$. In tal caso il musicista
terrà in considerazione solamente le probabilità di cambio $pc$ di un
record \emph{semiquaver} e deciderà quali note far suonare al player
basandosi sulle note contenute nell'accordo di rifermento che il direttore
gli ha comunicato. Di queste note creerà in maniera casuale una
qualche permutazione generando così dei rivolti e variazioni sulla
stessa struttura dell'accordo.\\
In linea generale in questa sezione sono state descritte le
caratteristiche principali del ragionamento di un musicista, fatta
eccezione per la modalità genetica, in cui le note non vengono
interamente improvvisate in tempo reale, ma subiscono un processo di 
mutazione genetica al fine di introdurre un metodo di apprendimento nel
modello. 
