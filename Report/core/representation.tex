\section{Rappresentazione della Conoscenza}
Generalmente la conoscenza musicale di ogni musicista, cantante,
compositore o direttore d'orchestra è formata da tre 
componenti fondamentali:
\begin{enumerate}
\item esecuzioni passate dell'artista stesso
\item esecuzioni altrui ascoltate precedentemente
\item regole provenienti dalla teoria musicale
\end{enumerate}

In IMprovEEsation queste informazioni sono alla base del modello della
conoscenza degli agenti del sistema. Le componenti 1 e 2 costituiscono i
pattern musicali e le relative note associate ad essi. Inoltre nella memoria
degli agenti sono presenti informazioni aggiuntive, come ad esempio
scale, modo degli accordi, etc. Quest'ultime vengono messe in relazione
con i pattern e le note formando così la terza componente, ovvero l'insieme 
di regole teoriche possedute dall'agente.
\subsection{Regole e Pattern}
% XXX MATTE XXX %
Definiamo i pattern come sequenze di misure di accordi. %TODO CHECK%
Il direttore ha la conoscenza di una collezione di diversi pattern che
a loro volta possono ammettere delle varianti di stile e genere.
%TODO il musicista conosce le note e in base alle relazioni tra accordi
%e scale e l'accordo inviato dal direttore decide..  %
\subsection{Database Relazionale}
% XXX MATTE XXX %
